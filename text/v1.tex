\section{Übersicht}
\begin{enumerate}
  \item Defintion Ladung, Strom, Spannung, Potential
  \item Grundstromrkeis, Darstellung
  \item Widerstand $\Omega$ Ohmisches Gesetz
  \item Mehrteilige Schaltungen
  \begin{enumerate}[label*=\arabic*.]
    \item Stromknoten
    \item Spannungsmasche
    \item Superpostionsprinzip
  \end{enumerate}
  \item Reale Spannungs und Stromquellen
  \item Leistung
  \item Zeit abhänige Spannugen und Ströme
  \item Blindelement $L,C$
\end{enumerate}


\section{Elektronen}

Elektronische Ladung $Q$ besteht aus \underline{Elektronnen}\\
Stets \underline{negative} Ladung \\
(Postive Ladung ist Elektronenmangel, setzt Stoff vorraus)\\
1 Elektron hat eine Elementarladung von $e = 1,6 \cdot 10^{-19} A \cdot s$\\
\\
\\
$ Q = A \cdot s $ ($A/s$  Amper Sekunde) wegen negative: $q = -1,6 \cdot 10^{-19} A \cdot s$\\
\\
$Q_{Elektron} = -1,6 \cdot 10^{-19} A \cdot s$\\
Für úberschus\\
\\
\\
Eine Wolke hat eine Ladung von $Q_{wolke} = 1 As$\\
\\
$ Q = n * q$\\
$n$ ist die Anzahl der $E$\\
$n = ? $\\
\\
$n = \frac{Q}{q} = \frac{1 As}{-1,6 \cdot 10^{-19} As}$\\
\\
Wir können kürzen:\\
$\frac{1 \cancel{As}}{-1,6 \cdot 10^{-19} \cancel{As}}$\\
$\approx \frac{1}{-1,6} \cdot 10^{19}$\\
$\approx \frac{5}{8} \cdot 10^{19}$\\
\\

\section{Spannung}
\underline{Spannung ist ein Zustand}\\
platzhalter bild\\
Es herrscht eine \underline{Spannung $U$}\\
Oder Wolke 2\\
Platzhalter bild2\\
$ U > 0 $\\
\\

Wolke3\\
Platzhalter Bild3\\

Spannungsrichtung festlegung mit einem Zählpfeile (\underline{willkührliche Annahme})\\
hier Spannung von Erde zur Wolke \underline{positive}\\
\\
Einheit der Spannung\\
Volt, kurz $V$\\
\\
Angabe ziB $U = + 230 V \leftarrow $ Buchstabe als Einheit\\

\newpage

Vorsätze

$ 1 =1 $\\
$ 1000 = 10^3 = k $ (kilo)\\
Übersicht:
$ 10^{-12} = p $ ("pico")\\
$ 10^{-9} = n $ ("nano")\\
$ 10^{-6} = \mu $ ("mikro")\\
$ 10^{-3} = m $ ("milli")\\
$ 10^{-2} = c $ ("centi")\\
$ 10^{-1} = d $ ("deci")\\
$ 10^{1} = da $ ("deca")\\
$ 10^{2} = h $ ("hecto")\\
$ 10^{3} = k $ ("kilo")\\
$ 10^{6} = M $ ("mega")\\
$ 10^{9} = G $ ("giga")\\
$ 10^{12} = T $ ("tera")\\:



